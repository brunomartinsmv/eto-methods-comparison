\documentclass[11pt,a4paper]{article}
\usepackage{geometry}
\usepackage{graphicx}
\usepackage{booktabs}
\usepackage{siunitx}
\usepackage{hyperref}
\geometry{margin=2.5cm}
\title{Estimativa da Evapotranspira\c{c}\~ao de Refer\^encia (ET$_0$) para Piracicaba em 2024: Compara\c{c}\~ao de M\'etodos e Calibra\c{c}\~ao Emp\'irica}
\author{Bruno et al.}
\date{2024}
\begin{document}
\maketitle
\begin{abstract}
Este estudo compara seis m\'etodos para estimativa da evapotranspira\c{c}\~ao de refer\^encia (ET$_0$) em Piracicaba (2024): Penman--Monteith (refer\^encia), Priestley--Taylor, Hargreaves--Samani, Thornthwaite, Thornthwaite--Camargo e Camargo. Os dados foram pr\'e-processados com reconstru\c{c}\~ao de datas e interpola\c{c}\~ao de faltantes. Avaliamos o desempenho di\'ario e mensal com m\'etricas (vi\'es, MAE, RMSE, $r$, $R^2$, NSE e \emph{d} de Willmott). Apresenta-se calibra\c{c}\~ao multiplicativa simples para reduzir vi\'es dos m\'etodos emp\'iricos. Os resultados indicam que Priestley--Taylor acompanha bem Penman--Monteith; Hargreaves e Camargo requerem calibra\c{c}\~ao local, ap\'os a qual atingem erros compar\'aveis aos m\'etodos f\'isicos em escala di\'aria.
\end{abstract}
\section{Metodologia}
Dados di\'arios de 2024 (arquivo Evapo\_2.xlsx, aba Piracicaba; altitude 546 m) foram padronizados. O balan\c{c}o de radia\c{c}\~ao l\'iquida $R_n$ foi usado diretamente quando dispon\'ivel e plaus\'ivel, ou estimado de Rs, Rso e Rnl (FAO-56). As f\'ormulas implementadas foram: PM (FAO-56 eq. 6), PT com $\alpha=1.26$, HS, TH (mensal com ajuste de dias), THC (TH ajustado por fotoper\'iodo) e Camargo ($ET=K\,Rs\,(T+20)$, $K=0.01$). O solo di\'ario considerou $G\approx0$.
\section{Resultados}
As Figuras referenciadas est\~ao na pasta do projeto: s\'eries 7d, dispers\~oes e res\'iduos (todas em 600 dpi). As Tabelas~\ref{tab:metricas_diarias} e \ref{tab:metricas_mensais} sumarizam o desempenho. Ap\'os calibra\c{c}\~ao, HS e Camargo alinharam o total anual ao PM e reduziram substancialmente RMSE/MAE.
\begin{table}
\caption{Métricas diárias (vs ET_PM) incluindo séries calibradas}
\label{tab:metricas_diarias}
\begin{tabular}{lrrrrrrrrr}
\toprule
 & n & bias & MAE & RMSE & r & R2 & NSE & slope & d_willmott \\
metodo &  &  &  &  &  &  &  &  &  \\
\midrule
ET_PT & 366 & 0.016000 & 0.461000 & 0.614000 & 0.914000 & 0.835000 & 0.797000 & 1.012000 & 0.953000 \\
ET_TH & 366 & -0.098000 & 0.980000 & 1.252000 & 0.423000 & 0.179000 & 0.156000 & 0.236000 & 0.592000 \\
ET_THC & 366 & -0.052000 & 0.990000 & 1.269000 & 0.439000 & 0.193000 & 0.133000 & 0.299000 & 0.648000 \\
ET_HS & 366 & 8.372000 & 8.372000 & 8.649000 & 0.834000 & 0.696000 & -39.276000 & 1.945000 & 0.257000 \\
ET_HS_CAL & 366 & -0.000000 & 0.611000 & 0.791000 & 0.834000 & 0.696000 & 0.663000 & 0.545000 & 0.869000 \\
ET_CAM & 366 & 4.604000 & 4.618000 & 4.914000 & 0.935000 & 0.874000 & -12.000000 & 2.005000 & 0.430000 \\
ET_CAM_CAL & 366 & -0.000000 & 0.335000 & 0.487000 & 0.935000 & 0.874000 & 0.872000 & 0.831000 & 0.963000 \\
\bottomrule
\end{tabular}
\end{table}

\begin{table}
\caption{Métricas mensais (somas por mês) vs ET_PM incluindo calibradas}
\label{tab:metricas_mensais}
\begin{tabular}{lrrrrrrr}
\toprule
 & n & bias & MAE & RMSE & r & R2 & d_willmott \\
metodo &  &  &  &  &  &  &  \\
\midrule
ET_HS_CAL & 12 & -0.000000 & 5.290000 & 5.960000 & 0.958000 & 0.917000 & 0.978000 \\
ET_CAM_CAL & 12 & -0.000000 & 6.474000 & 7.752000 & 0.927000 & 0.860000 & 0.962000 \\
ET_TH & 12 & -2.988000 & 9.631000 & 12.692000 & 0.842000 & 0.709000 & 0.908000 \\
ET_THC & 12 & -1.586000 & 11.057000 & 14.184000 & 0.875000 & 0.765000 & 0.909000 \\
ET_PT & 12 & 0.480000 & 12.518000 & 14.236000 & 0.835000 & 0.698000 & 0.900000 \\
ET_CAM & 12 & 140.419000 & 140.419000 & 143.636000 & 0.927000 & 0.860000 & 0.224000 \\
ET_HS & 12 & 255.333000 & 255.333000 & 260.881000 & 0.958000 & 0.917000 & 0.136000 \\
\bottomrule
\end{tabular}
\end{table}

\section{Discuss\~ao}
PT apresentou bom ajuste pela domin\^ancia de $R_n$ no controle da ET$_0$ em 2024. TH/THC reproduzem sazonalidade, mas n\~ao a variabilidade di\'aria, sendo adequados para climatologia. HS e Camargo superestimam com coeficientes gen\'ericos; a calibra\c{c}\~ao multiplicativa removeu o vi\'es e melhorou substancialmente as m\'etricas sem comprometer correla\c{c}\~ao. Recomenda-se valida\c{c}\~ao cruzada em anos distintos e uso operacional de PM (ou PT) quando poss\'ivel.
\section{Conclus\~oes}
- PM permanece como refer\^encia; PT \`e alternativa parcimoniosa.
- HS e Camargo exigem calibra\c{c}\~ao local para uso pr\'atico.
- TH/THC s\~ao \'uteis para estudos mensais de clima, menos para manejo di\'ario.
\paragraph{Dados e c\'odigo} Arquivos gerados incluem CSV consolidado e figuras (600 dpi).\newline
\textbf{Reprodutibilidade:} notebook calculo2.ipynb.
\end{document}
